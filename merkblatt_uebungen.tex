\documentclass[a4paper,oneside,11pt,DIV=14]{scrartcl}
\usepackage{merkblatt}
\usepackage{blindtext}
\DeclareOldFontCommand{\bf}{\normalfont\bfseries}{\mathbf}

\newcommand{\farbe}{1} %Farbiges Uni-Logo? ja=1, nein=0
\newcommand{\fakult}{Mathematisch-Naturwissenschaftliche Fakultät}	%Fakultät
%\newcommand{\fbname}{Wilhelm-Schickard-Institut für Informatik - Fachbereich Visual Computing}	%Fachbereich oder Name des Instituts
\newcommand{\fbname}{WSI / Visual Computing}	%Fachbereich oder Name des Instituts
\newcommand{\kursname}{3D Computer Vision}	%Vorlesungs-/Klausurtitel
\newcommand{\semester}{Sommersemester 2020}	%Semester
\newcommand{\blatt}{Merkblatt}
\newcommand{\dozent}{Prof. A. Schilling}
\newcommand{\tutor}{Dipl.-Inf. M. Lange}

\begin{document}
\deckblatt

\section*{Übungen}

Übungsblätter sind wie die Vorlesungsfolien im Ilias-System hinterlegt:\\ 
{Mathematisch-Naturwissenschaftliche Fakultät » Informatik » Visual Computing » INF4161 3D Computer Vision (Bildverarbeitung II)}\
%\\Passwort: multiview\\
\\\\
Kriterien für das Bestehen der Übungen:
\begin{itemize}
\item Alle Blätter sind zu bearbeiten.
\item Es gilt die Pflicht zur digitalen Anwesenheit für \textbf{alle} Übungsteilnehmer in der Übungsgruppe bzw. bei den Abgaben. In Krankheitsfällen bitte \textbf{vorher} den Tutor informieren. Sonstige Ausnahmen nur nach Absprache.
\item Es müssen insgesamt 50\% der möglichen Punkte erreicht werden.
\item In der Übungsgruppe bzw. bei der Abgabe muss \textbf{jeder} Übungsteilnehmer \textbf{jede} Aufgabe erklären können.
%\item Jeder Übungsteilnehmer muss mind. 1 Teilaufgabe vorgerechnet/vorgetragen haben.
\item Die bearbeiteten Aufgaben sind vor Ablauf der Frist über das Ilias-System abzugeben.
\item Plagiate werden mit null Punkten gewertet und können zum Ausschluss aus der Vorlesung führen. Jedes Teammitglied ist für alle abgegebenen Aufgaben verantwortlich.
\end{itemize}

\section*{Hinweise}
\begin{itemize}
\item Für jedes Übungsblatt gibt es ein separates Übungsobjekt.
\item Die Lösungen müssen innerhalb des Übungsobjekts von einer Person des Teams als Team abgegeben werden.
\item Jedes Team besteht aus 3 Personen. Ausnahmen sind nur nach Absprache mit den Tutoren möglich.
\item Nach Ablauf der Abgabefrist ist es nicht mehr möglich Lösungen abzugeben.
\end{itemize}

\end{document}
